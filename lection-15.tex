\documentclass[aspectratio=169]{beamer}
\usepackage[utf8]{inputenc}
\usepackage[english,russian]{babel}
\usepackage{cancel}
\usepackage{amssymb}
\usepackage{stmaryrd}
\usepackage{cmll}
\usepackage{graphicx}
\usepackage{amsthm}
\usepackage{tikz}
\usepackage{multicol}
\usetikzlibrary{patterns}
\usepackage{chronosys}
\usepackage{proof}
\usepackage{multirow}
\usepackage{comment}
\setbeamertemplate{navigation symbols}{}
%\usetheme{Warsaw}

\newtheorem{thm}{Теорема}[section]
\newtheorem{dfn}{Определение}[section]
\newtheorem{lmm}{Лемма}[section]
\newtheorem{exm}{Пример}[section]
\newtheorem{snote}{Пояснение}[section]

\newcommand{\divisible}%
{\mathrel{\lower.2ex%
\vbox{\baselineskip=0.7ex\lineskiplimit=0pt%
\kern6pt \hbox{.}\hbox{.}\hbox{.}}%
}}

\begin{document}

\newcommand\doubleplus{+\kern-1.3ex+\kern0.8ex}
\newcommand\mdoubleplus{\ensuremath{\mathbin{+\mkern-10mu+}}}

\begin{frame}{}
\begin{center}\Large Метод резолюций \end{center}
\end{frame}
\begin{comment}
\begin{frame}{Компактность}
\begin{dfn} 
Пространство $X$ компактно, если из любого его открытого покрытия $U$ можно выделить конечное подпокрытие:

$X = \cup U$, существует $V \subseteq U$, что $|V| < \aleph_0$ и $X = \cup V$.
\end{dfn}

\begin{exm}
$(0,1)$ не компактен. Например, $U = \{(\varepsilon/2,\varepsilon)\ |\ \varepsilon\in(0,1)\}$.
Пусть $V \subset U$ и $|V| < \aleph_0$. Тогда $\min \{ a\ |\ (a,b) \in V \} > 0$.
\end{exm}

\begin{exm}
$[0,1]$ компактен. Выберем $U$ и покажем, что в нём есть подпокрытие.
Рассмотрим все подотрезки вида $[a,x]$ где $a < x$, имеющие конечное покрытие.
Несложно показать, что $\max x = 1$.
\end{exm}
\end{frame}

\begin{frame}{Фильтры}
\begin{dfn}
Фильтром $\mathcal{F}$ назовём структуру на элементах 
некоторой булевой алгебры $\langle L, (\preceq) \rangle$ со следующими свойствами:
\begin{itemize}
\item $\mathcal{F} \ne \varnothing$, $0 \notin \mathcal{F}$;
\item если $a,b \in \mathcal{F}$, то $a \cdot b \in \mathcal{F}$;
\item если $a \in \mathcal{F}$, $a \preceq b$, $b \in L$, то $b \in \mathcal{F}$.
\end{itemize}
\end{dfn}

\begin{exm}
Главный фильтр для множества: $\{ E\subseteq U\ |\ S \subseteq E \}$.
\end{exm}
\begin{exm}
Фильтр Фреше: $\{ X\subseteq U\ |\ |cX| < \aleph_0\}$ при $|U| \ge \aleph_0$.
\end{exm}
\end{frame}

\begin{frame}{База и предбаза фильтра}
\begin{dfn}База фильтра --- семейство непустых множеств, что
любое конечное пересечение множеств также принадлежит базе.\end{dfn}
\begin{dfn}Предбаза фильтра --- семейство непустых множеств, что
любое конечное пересечение множеств непусто.\end{dfn}
\begin{dfn}Если $B$ --- предбаза фильтра, то $\{ X \subseteq U\ |\ X\}$\end{dfn}
\end{frame}

\begin{frame}{Ультрафильтры}
\begin{dfn}
Ультрафильтр $\mathcal{U}$ --- фильтр, не являющийся собственным подфильтром никакого 
другого фильтра. Нет $\mathcal{F}$, что $\mathcal{U}\subset\mathcal{F}$.
\end{dfn}

\begin{exm}
$\{ X \subseteq \mathbb{R}\ |\ 1 \in X \}$ --- ультрафильтр.
\end{exm}

\begin{thm}Фильтр $\mathcal{F}$ над $L$ --- ультрафильтр тогда и только тогда,
когда для всех $x \in L$ либо $x \in \mathcal{F}$, либо $\sim x \in \mathcal{F}$.
\end{thm}
\begin{proof}$(\Rightarrow)$.
Пусть $x \notin \mathcal{F}$ и $\sim x \notin \mathcal{F}$. Тогда $\mathcal{F}\cup\{x\}$ --- предбаза
фильтра, так как если $p \in \mathcal{F}$ и $p \cdot x = 0$, то $p \cdot x + \sim x = \sim x$
и далее $\sim x = (p + \sim x)\cdot (x + \sim x) = p + \sim x$. Поэтому $p \preceq \sim x$,
отчего $\sim x \in \mathcal{F}$.

$(\Leftarrow)$. Если есть фильтр больше, то обязательно для какого-то $t$ и $t \in \mathcal{G}$,
и $\sim t \in \mathcal{G}$, но тогда $t \cdot \sim t = 0$.
\end{proof}
\end{frame}

\begin{frame}{Произведение интерпретаций}
Пусть есть две похожие математические структуры. Например, $\langle A, \prec \rangle$
и $\langle B, \prec \rangle$.
Расширим это свойство на пары: $\langle a,b \rangle \prec \langle c,d\rangle$, если
$a \prec c$ и $b \prec d$. 

\begin{dfn}Пусть даны интерпретации $M_i$ предиката $P$ 
Положим $\Pi_i M_i \models P(x_1,\dots,x_n)$, если $M_k \models P(x_1,\dots,x_n)$
при всех $M_k$.
\end{dfn}

Просто, но не всегда работает: линейный порядок можно потерять:
$M_1, M_2$ --- модели с $D = \mathbb{N}$, тогда $M_1 \times M_2 \models x_1 \le x_2$ только если
$M_1 \models x_1 \le x_2$ и $M_2 \models x_1 \le x_2$ (то есть отношение имеет место покомпонентно).

\begin{dfn}Пусть даны интерпретации $M_i$ предиката $P$ и фильтр $F$.
Положим $\Pi_i M_i/F \models P(x_1,\dots,x_n)$, если $\{ k \ |\ M_k \models P(x_1,\dots,x_n) \} \in F$
(назовём это фильтрованным произведением интерпретаций).
\end{dfn}

Если $F$ --- ультрафильтр, то порядок останется линейным.
\end{frame}

\begin{frame}{Теорема Лося}
Доопределим связки на $\Pi_i M_i/F$ естественным образом.
\begin{thm}
Пусть $M_i$ --- семейство интерпретаций и $U$ --- некоторый ультрафильтр на индексном множестве семейства.
$\Pi_i M_i/U\models \varphi$ тогда и только тогда, когда $\{ k \ |\ M_k \models \varphi\} \in U$.
\end{thm}
\begin{proof}Индукция по структуре\end{proof}
\end{frame}

\begin{frame}{Топология на полных непротиворечивых множествах формул}
\begin{dfn}Пусть $\varphi$ --- формула языка $L$. Сопоставим ей $\langle \varphi\rangle$ --- множество
всех полных непротиворечивых множеств формул $\Gamma$, что $\Gamma \vdash \varphi$. 
\end{dfn}
\begin{thm}$B = \{\langle\varphi\rangle\ |\ \varphi\in L\}$ образует базу топологии.\end{thm}
Назовём это пространство $\mathcal{L}$.
\begin{thm}Замкнутое множество \end{thm}
\end{frame}
\end{comment}

\begin{frame}{Ищем доказательство в исчислении предикатов}
Хотим научиться проверять доказуемость формул исчисления предикатов.
В общем случае невозможно, но человек как-то справляется? Может,
для каких-то частных случаев мы сможем предложить метод? 

По теореме о полноте можем рассматривать $(\models)$ вместо $(\vdash)$. 
Напомним: $\models \alpha$, если для всех $M = \langle D, F, P, E \rangle$ выполнено $M \models \alpha$.
Нам мешает:
\begin{enumerate}
\item бесконечное множество предметных множеств $D$ и оценок;
\item бесконечный перебор для кванторов;
\end{enumerate}
Будем последовательно упрощать задачу:
\begin{enumerate}
\item упростим формулу;
\item заменим произвольное $D$ на рекурсивно-перечислимое, устроенное некоторым фиксированным образом;
\item научимся по этому перечислимому $D$ искать доказательство / противоречие.
\end{enumerate}
\end{frame}

\begin{frame}{Компактность}
\begin{dfn} 
Пространство $X$ компактно, если из любого его открытого покрытия $U$ можно выделить конечное подпокрытие:

$X = \cup U$, существует $V \subseteq U$, что $|V| < \aleph_0$ и $X = \cup V$.
\end{dfn}

\begin{exm}
$(0,1)$ не компактен. Например, $U = \{(\varepsilon/2,\varepsilon)\ |\ \varepsilon\in(0,1)\}$.
Пусть $V \subset U$ и $|V| < \aleph_0$. Тогда $\min \{ a\ |\ (a,b) \in V \} > 0$.
\end{exm}

\begin{exm}
$[0,1]$ компактен. Выберем $U$ и покажем, что в нём есть подпокрытие.
Рассмотрим все подотрезки вида $[a,x]$ где $a < x$, имеющие конечное покрытие.
Несложно показать, что $\max x = 1$.
\end{exm}
\end{frame}


\begin{frame}{Теорема Гёделя о компактности}
%\begin{thm}Пространство $\mathcal{L}$ --- вполне несвязно компактное.
%\end{thm}
\begin{thm}Если $\Gamma$ --- некоторое семейство формул, то $\Gamma$ имеет модель
тогда и только тогда, когда любое его конечное подмножество имеет модель.\end{thm}
%\begin{proof}Рассмотрим все полные семейства формул, содержащие $\Gamma$.
%\end{proof}
\end{frame}

\begin{frame}{Сколемизация. Упрощаем формулу.}
\begin{enumerate}
\item Предварённая форма (поверхностные кванторы):
$\psi := Q x_1.Q x_2\dots Q x_n.\varphi(x_1,\dots,x_n)$

\item Для упрощения предполагаем, что кванторы чередуются.
Это не сильно уменьшает общность. Например, если $D = \mathbb{N}$,
то $(\forall x.\forall y.\varphi(x,y)) \leftrightarrow (\forall p.\varphi(\text{plog}_2(p),\text{plog}_3(p))$

\item Убрать кванторы существования:
$\zeta = \exists x_1.\forall x_2.\exists x_3.\forall x_4\dots\exists x_{n-1}.\forall x_n.\varphi(x_1,\dots,x_n)$

Заменим $x_{2k+1}$ функцией Сколема: $e_{2k+1}(x_2,x_4,\dots,x_{2k})$.

Получим: $\eta = \forall x_2.\forall x_4\dots\forall x_n.\varphi(e_1,x_2,e_3(x_2),\dots,e_{n-1}(x_2,x_4,\dots,x_{n-2}),x_n)$
Очевидно, что $\vdash\zeta$ эквивалентно $\models\zeta$ эквивалентно $\models\eta$ и $\vdash\eta$.

\item ДНФ:
$$\forall x_2.\forall x_4\dots\forall x_n.\bigwedge_c\left(\bigvee_{i = \overline{1,d(c)}} (\neg)P_i(\theta_i)\right)$$

\end{enumerate}
\end{frame}

\begin{frame}{Эрбранов универсум}
\begin{dfn}Пусть $H_0(\varphi)$ --- все константы в формуле $\varphi$ (либо особая константа $a$, если констант в $\varphi$ нет)\\
$H_1(\varphi)$ --- $H_0(\varphi)$ и все функции от значений $H_0(\varphi)$ (как строки)\\
$H_2(\varphi)$ --- $H_1(\varphi)$ и все функции от значений $H_1(\varphi)$ (как строки)

$H = \cup H_n(\varphi)$ --- основные термы.
\end{dfn}

\begin{exm}$P(a)\vee Q(f(b))$: \\
$H_0 = \{a,b\}$\\
$H_1 = \{a,b,f(a),f(b)\}$\\
$H_2 = \{a,b,f(a),f(b),f(f(a)),f(f(b))\}$\\
$\dots$\\
$H = \{f^{(n)}(x)\ |\ n \in \mathbb{N}_0, x \in \{a,b\}\}$\end{exm}
\end{frame}

\begin{frame}{Выполнимость}
\begin{thm}Формула выполнима тогда и только тогда, когда она выполнима на Эрбрановом универсуме.\end{thm}
\begin{proof}
$(\Rightarrow)$ Пусть $M \models\forall \overline{x}.\varphi$. Тогда построим отображение $\text{eval}: H \rightarrow M$
(смысл названия вдохновлён языками программирования: $\text{eval}(``f(f(b))``)$ перейдёт в $f(f(b))$, где $f$ и $b$ --- из $M$).

Предикатам дадим согласованную оценку:
$P_H(t_1,\dots,t_n) = P_M(h(t_1),\dots,h(t_n))$. Очевидно, любая формула сохранит своё значение, кванторы всеобщности
по меньшему множеству также останутся истинными.

$(\Leftarrow)$ Очевидно.
\end{proof}\end{frame}

\begin{frame}{Противоречивость}
\begin{dfn}Система дизъюнктов $\{\delta_1,\dots,\delta_n\}$ противоречива,
если для каждой интерпретации $M$ найдётся $\delta_k$ и такой набор $d_1\dots d_v$,
что $\llbracket\delta_k\rrbracket^{x_1 := d_1, \dots, x_v := d_v} = \text{Л}$.\end{dfn}
\begin{thm}Система дизъюнктов противоречива, если она невыполнима на Эрбрановом универсуме.\end{thm}
\begin{proof}Контрапозиция теоремы о выполнимости + разбор определения.
\end{proof}
\end{frame}

\begin{frame}{Основные примеры}
\begin{dfn}
Дизъюнкт с подставленными основными термами вместо переменных называется основным примером.
Системой основных примеров назовём множество основных примеров опровержимых дизъюнктов:

Если $M \not\models \delta_k$ для некоторой эрбрановской интерпретации, то
возьмём все возможные основные примеры $\delta_k$.
\end{dfn}

\begin{thm}Система дизъюнктов $S$ противоречива тогда и только тогда, когда система всевозможных
основных примеров $E$ противоречива\end{thm}
\begin{proof}Для некоторой эрбрановой интерпретации дизъюнкт $\delta_k$
опровергается тогда и только тогда, когда соответствующая ему подстановка в $E$ опровергается.
\end{proof}\end{frame}

\begin{frame}{Теорема Эрбрана}
\begin{thm}[Эрбрана]Система дизъюнктов $S$ противоречива тогда и только тогда, когда существует
конечное противоречивое множество основных примеров системы дизъюнктов $S$\end{thm}
\begin{proof}$(\Leftarrow)$ 
Пусть $\delta_1[\overline{x} := \overline{\theta}],\dots,\delta_k[\overline{x} := \overline{\theta}]$ 
--- противоречивое множество примеров дизъюнктов. Тогда интерпретация $\overline{\theta}$
опровергает хотя бы один из $\delta_k$ и система противоречива.

$(\Rightarrow)$ Если $S$ противоречива, то значит, множество основных примеров $S$
противоречиво (по теореме о выполнимости Эрбранова универсума). Тогда по теореме компактности
в нём найдётся конечное противоречивое подмножество.
\end{proof}
\end{frame}

\begin{frame}{Правило резолюции (исчисление высказываний)}
Пусть даны два дизъюнкта, $\alpha_1 \vee \beta$ и $\alpha_2 \vee \neg \beta$.
Тогда следующее правило вывода называется правилом резолюции:

$$\infer{\alpha_1\vee \alpha_2}{\alpha_1\vee \beta\quad\quad \alpha_2\vee\neg \beta}$$

\begin{thm}Система дизъюнктов противоречива, если в процессе всевозможного применения
правила резолюции будет построено явное противоречие, 
т.е. найдено два противоречивых дизъюнкта: $\beta$ и $\neg\beta$.
\end{thm}
\end{frame}

\begin{frame}{Алгебраические термы}
	\begin{dfn}Алгебраический терм $$\theta := x\:|\:(f(\theta_1,\ldots,\theta_n))$$ 
где $x-$переменная, $f(\theta_1,\ldots,\theta_n)-$применение функции. Напомним, что константы --- нульместные
функциональные символы, собственно переменные будем обозначать последними буквами латинского алфавита. \end{dfn}
	%\subsection{Уравнение в алгебраических термах $\theta_1=\theta_2$\\Система уравнений в алгебраических термах}
	\begin{dfn}Система уравнений в алгебраических термах
	$
		\begin{cases}
			\theta_1=\sigma_1&\\
			\vdots&\\
			\theta_n=\sigma_n&\\
		\end{cases}
	$\par где $\theta_i \text{ и } \sigma_i-\text{термы}$\par
\end{dfn}
\end{frame}
\begin{frame}{Уравнение в алгебраических термах}
	\begin{dfn}$\{x_i\}=X-$множество переменных, $\{\theta_i\}=T-$множество термов.\end{dfn}
	\begin{dfn}Подстановка$-$отображение вида: $\pi_0:X\to T$, тождественное почти везде.
        \par $\pi_0(x)$ может быть либо $\pi_0(x)=\theta_i\text{, либо }\pi_0(x)=x$.\end{dfn} 
	Доопределим $\pi:T\to T$, где \begin{enumerate}
		\item $\pi(x)=\pi_0(x)$
		\item $\pi(f(\theta_1, \ldots, \theta_k))=f(\pi(\theta_1), \ldots, \pi(\theta_k))$
	\end{enumerate}
	
	\begin{dfn}Решить уравнение в алгебраических термах$-$найти такую наиболее общую подстановку $\pi$, 
        что $\pi(\theta_1)=\pi(\theta_2)$.
Наиболее общая подстановка --- такая, для которой другие подстановки являются её частными случаями.\end{dfn} 
\end{frame}

\begin{frame}{Задача унификации}
\begin{dfn}
Пусть даны формулы $\alpha$ и $\beta$. Тогда решением задачи унификации
будет такая наиболее общая подстановка $\pi = \mathcal{U}\big[\alpha,\beta\big]$, что $\pi(\alpha) = \pi(\beta)$.

%Будем писать $\Sigma = \mathcal{U}\[\alpha,\beta\]$, если $\Sigma(\alpha) = \Sigma(\beta) = \eta$ и $\Sigma$ --- наиболее общая.
Также, $\eta$ назовём наиболее общим унификатором.
\end{dfn}

\begin{exm}
\begin{itemize}
%\item $\mathcal{U}\[ P(x,g(b)),P(f(a),y) \] = [ x := f(a), y := f(b) ]$ 
%и $P(f(a),g(b))$ --- унификатор.
\item Формулы $P(a,g(b))$ и $P(c,d)$ не имеют унификатора (мы считаем, что $a,b,c,d$ --- нульместные функции, а
$f$ --- одноместная функция).
\end{itemize}
\end{exm}
\end{frame}

\begin{frame}{Правило резолюции для исчисления предикатов}
\begin{dfn}
Пусть $\sigma_1$ и $\sigma_2$ --- подстановки, заменяющие переменные в формуле на свежие. 
Тогда правило резолюции выглядит так:

$$\infer[{\pi = \mathcal{U}\big[\sigma_1(\beta_1),\sigma_2(\beta_2)\big]}]
        {\pi(\sigma_1(\alpha_1)\vee \sigma_2(\alpha_2))}
        {\alpha_1\vee \beta_1\quad\quad\alpha_2\vee\neg \beta_2}$$
\end{dfn}

$\sigma_1$ и $\sigma_2$ разделяют переменные у дизъюнктов, чтобы $\pi$ не осуществила лишние
замены, ведь $\vdash(\forall x.P(x) \with Q(x)) \leftrightarrow (\forall x.P(x))\with(\forall x.Q(x))$, но
$\not\vdash (\forall x.P(x) \vee Q(x)) \rightarrow (\forall x.P(x))\vee(\forall x.Q(x))$.

\begin{exm}\vspace{-0.5cm}
$$\infer[\text{ подстановки: } \sigma_1(x) = x', \sigma_2(x) = x'', \pi(x')=a]{Q(a)\vee T(x'')}{Q(x)\vee P(x) \quad \neg P(a)\vee T(x)}$$
\end{exm}
\end{frame}

\begin{frame}{Метод резолюции}
Ищем $\vdash\alpha$.

\begin{enumerate}
\item найдём опровержение $\neg\alpha$.
\item перестроим $\neg\alpha$ в ДНФ.
\item будем применять правило резолюции, пока получаем новые дизъюнкты и пока 
не найдём явное противоречие (дизъюнкты вида $\beta$ и $\neg\beta$).
\end{enumerate}

Если противоречие нашлось, значит, $\vdash\neg\neg\alpha$. Если нет --- значит, $\vdash\neg\alpha$.
Процесс может не закончиться.
\end{frame}

\begin{frame}{SMT-решатели}

Обычно требуется не логическое исчисление само по себе, а теория первого порядка.
То есть, <<Satisfability Modulo Theory>>, <<выполнимость в теории>> --- вместо SAT, выполнимости.
\begin{itemize}
\item Иногда можно вложить теорию в логическое исчисление, 
даже в исчисление высказываний: $\overline{S_2S_1S_0} = \overline{A_1A_0}+\overline{B_1B_0}$
$$\begin{array}{ll}
S_0 = A_0 \oplus B_0 & C_0 = A_0 \with B_0\\
S_1 = A_1 \oplus B_1 \oplus C_0 & C_1 = (A_1 \with B_1) \vee (A_1 \with C_0) \vee (B_1 \with C_0) \\
S_2 = C_1\end{array}$$

\item А можно что-то добавить прямо на уровень унификации / резолюции:
Например, можем зафиксировать арифметические функции --- и производить вычисления
в правиле резолюции вместе с унификацией.

Тогда противоречие в $\{x = 1+3+1,\neg x = 5\}$ можно найти за один шаг.
\end{itemize}
\end{frame}

\end{document}
